\documentclass[a4paper, UTF8]{ctexart}

\usepackage{amsmath}
\usepackage{amssymb}
\usepackage{graphicx}
\usepackage[colorlinks=false]{hyperref}
\usepackage{listings}
\usepackage{xcolor}
\newtheorem{proposition}{命题}
\DeclareMathOperator{\sgn}{\mathrm{sgn}}
\DeclareMathOperator{\trunc}{\mathrm{trunc}}
\lstset{
    basicstyle=\small\tt,
    commentstyle=\color{gray},
    numbers=left
}

\title{将一浮点数转换为两根式之和的最简算法}
\author{郑}

\begin{document}
\maketitle

\section{引言}
为了减少重复度,我们定义一函数\[\varphi(x,y)=\sgn(x)\sqrt{|x|}+\sgn(y)\sqrt{|y|} \quad x,y\in\mathbb{Z}\]若仅知浮点数$n=\varphi(a,b)$,则可以使用一些数学技巧并仅仅使用一个while循环来求出$a$和$b$,而不是使用双重for循环。简而言之,我们是要找出$\varphi^{-1}(x)$。

\section{理论}
首先,使用既定的公式计算$\varphi^{-1}(x)$是不可能的,浮点数已经损失了太多的信息。使用穷举法便成为了唯一的方法,不过使用何种方法以及从何处开始搜寻时一个值得讨论的问题。前者我已经在引言处说明,接下来要讨论的是后者,请先看如下命题
\begin{proposition}
    有两数之和$a+b$,且$c=\dfrac{a+b}{2}$,那么$|c-a|=|c-b|$。
\end{proposition}
该命题毫无疑问是真的,证明就不给出了。根据该命题,有$n=\varphi(a,b)$,我们可以从$\dfrac{n}{2}$处往$+\infty$或$-\infty$方向搜索,只要确定了$a$就可以有唯一的$b$,再进行一次条件判断就可以确定它们是否为所求。

当然,以$\dfrac{n}{2}$作为起始值不太妥当,我们要求的是两个整数,以截尾取整后的$\left(\dfrac{n}{2}\right)^2$作为起始值更合适。

不过,考虑$\left(\dfrac{\sqrt{100}+\sqrt{101}}{2}\right)^2\approx100.499$,截尾取整后为100,这样的设计亦有疏漏。但是,有如下极限\[\lim_{x\to+\infty} \left(\frac{\sqrt{x}+\sqrt{x+1}}{2}\right)^2-x=\frac{1}{2}\]这说明函数$f(x)=\left(\dfrac{\sqrt{x}+\sqrt{x+1}}{2}\right)^2$可近似成$x+\dfrac{1}{2}$,而且当$x>5.76$时误差小于$1\%$;$x>62.001$时误差小于$0.1\%$。

以上,我们可以确定起始值
\[
    start(n)=
    \begin{cases}
        \trunc(\frac{n}{2})^2+\frac{1}{2} & \text{if }n>0 \\
        -\trunc(\frac{n}{2})^2-\frac{1}{2} & \text{if }n<0
    \end{cases}
\]

以下为了书写方便,将$\sqrt{n}$定义为$\sgn(n)|n|^{1/2}$。

已知$n=\varphi(a,b)$,在确定了起始值$start$之后,令$step=0.5$,找到一个端点\[a'=start-step\]然后可计算另一端点\[b'=\frac{n}{2}-|\sqrt{a}-\frac{n}{2}|\]由于$b'$必须为整数,故还需做如下处理\[b'=\sqrt{\sgn(b')\mathrm{round}(b'^2)}\]其中round表示无条件进位。

如果$\sqrt{a'}+\sqrt{b'}=n$的话,那么$a'$和$b'$就是要求的$a$和$b$。不然的话使$step\pm1$,向正无穷或负无穷方向继续寻找。

\section{实现}
这里给出的示例使用的是python标准库,也可以使用mpmath等库来获得更高的精度。
\begin{lstlisting}[language=python, name=example]
import math

def _fsqrt(n):
    return math.copysign(math.sqrt(math.fabs(n)), n)

def num2sqrts(n, max_num=1000):
    if n >= 0:
        mid = math.floor((n / 2) ** 2) + 0.5
    elif n < 0:
        mid = math.ceil(-(n / 2) ** 2) - 0.5
    actual_mid = n / 2
    t = 0.5
    while True:
        a = _fsqrt(mid + t)
        d = math.fabs(a - actual_mid)
        b = actual_mid - d
        b = _fsqrt(math.copysign(round(b ** 2), b))
\end{lstlisting}
理论上,这个算法可以一直运行下去,直到找到合适的值。但是考虑到实际情况,我还是设置了停止条件。
\begin{lstlisting}[language=python, name=example]
        if abs(a ** 2) > max_num or abs(b ** 2) > max_num:
            return
        if math.isclose(a + b, n):
            return int(round(math.copysign(a ** 2, a))), \
                int(round(math.copysign(b ** 2, b)))
        t += 1
\end{lstlisting}
最后,函数num2sqrts返回的是长度为2的元组,若返回值为\verb|None|则说明没有找到解析解。

\begin{figure}[htb]
	\centering
	\includegraphics[width=0.8\linewidth]{perform.pdf}
	\caption{算法的表现}
	\label{fig:perform}
\end{figure}

为了说明这个算法的表现,使$n=\varphi(x,y)$,计算求出$x$和$y$需要几次循环,便绘制了图\ref{fig:perform}。可见,当$x$和$y$的差值越大时,就需要更长的时间。

\section{结论}

\end{document}
